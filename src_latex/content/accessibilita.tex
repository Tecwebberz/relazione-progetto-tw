\section{Accessibilità}
Tutto lo sviluppo del sito si è svolto tenendo a mente le raccomandazioni dello standard WCAG 2.0.

\subsubsection{Separazione tra contenuto, presentazione e struttura}
La separazione tra queste tre parti fondamentali del sito ha permesso di gestire al meglio le possibili categorie di accesso effettuate dai vari utenti. La parte di contenuto è stata sviluppata tramite HTML5 in modo da sfruttare a pieno i tag semantici e le nuove aggiunte dello standard. Tramite CSS sono state poi aggiunte tutte le regole di presentazione per il layout del sito. Il comportamento dinamico del sito è stato sviluppato con il linguaggio Javascript.\\ Sfruttando gli strumenti del W3C, ad esempio il validatore di HTML e CSS, ci siamo accertati di aver rispettato tutte le regole dello standard.

\subsubsection{Assenza di CSS}
In caso di assenza di CSS, il sito mantiene una struttura comprensibile e veicola tutto il contenuto esclusivamente in HTML5. Per un'utente che utilizza strumenti di accesso facilitato per visitare il sito tutto è comunque raggiungibile.

\subsubsection{Assenza di JS}
Gli script Javascript sfruttano tutte funzionalità base del linguaggio e quindi non richiedono librerie esterne.

\subsubsection{Attributi ARIA}
Nonostante aver utilizzato il più possibile i tag semantici offerti da HTML5 abbiamo inserito anche alcuni attributi ARIA in modo da rendere più "espressivi" e accessibli.
\begin{itemize}
    \item \textbf{Breadcrumb:} la braedcrumb è stata definita tramite \textit{aria-label} come \textit{breadcrumb}, inoltre la pagina attuale è indicata dall'attributo \textit{aria-page="current"}.
    \item \textbf{Validazione dei form:} lo script di validazione del form aggiunge un elemento nel DOM dinamicamente. Viene quindi indicato tramite dal ruolo \textit{aria-role="alert"} che porta l'attenzione dello screen reader su di esso quando questoviene inserito nella pagina.
    \item \textbf{Bottoni link:} quando non è stato possibile sfruttare tag come \textit{<button>} è stato necessario usare l'attributo \textit{role="button"} su quei link \textit{<a>} che vengono rappresentati come bottoni via CSS.
    \item \textbf{Link aule:} nelle card delle aule è stata aggiunta una \textit{aria-label} dinamica per dare maggior significato al link \textit{Più info}.´
    \item \textbf{Sistema di rating:} la struttura del sistema di rating a stelle fa affidamento al foglio di stile per creare l'animazione delle stelle. In caso di assenza di esso, tutto il form mantiene una struttura accessibile: tutti gli input hanno una label identificativa univoca. Le \textit{<i>} necessarie alle stelle sono state dichiarate \textit{aria-disabled} per gli screenreader.
\end{itemize}
Altri attributi aria come \textit{aria-requried} e simili nei form sono stati omessi dato che sono già presenti quelli nativi di HTML5 che veicolano lo stesso significato.

\subsubsection{Contrasti}
Si è preso come principale colore uno che rimandi all'ateneo e al dipartimento. Da lì tutti gli altri colori sono stati scelti in modo da avere un contrasto che rispetti gli standard AAA del WCAG sfruttando lo strumento online \href{https://app.contrast-finder.org/}{Contrast Finder}. Queste analisi sono state fatte tramite il sito \href{https://contrastchecker.com/}{WCAG - Contrast Checker} e i \textit{Firefox Developer Tools}.

\subsubsection{Tabindex}
Non è sembrato necessario modificare l'ordine dei tabindex manualmente. Per come è stato strutturato il sito questi sono già organizzati in modo adeguato.

\subsubsection{Lingue straniere e termini abbreviati}
Il sito specifica come lingua principale l'italiano. Ogni parola che deve essere letta con pronuncia straniera è stata contrassegnata tramite l'attributo \textit{lang} in un tag di comodo come \textit{span} oppure nel suo tag di appartenza. Ad esempio la semplice parola \textit{Home} nella breadcrumb oppure \textit{stage} nelle FAQ. Inoltre tutte le sigle utilizzate (ad esempio \textit{ESU} ed \textit{ISEE}) sono state inserite all'interno di dei tag \textit{<abbr>}.

\subsection{Strumenti di testing}

Tutte le pagine e i fogli di stile sono state controllate tramite il \href{https://validator.w3.org/}{validatore HTML e CSS} offerto dal W3C.\\
Durante tutto il ciclo di sviluppo la maggior parte dei controlli è stata fatta tramite i \textit{Firefox Developer Tools} e il software \textit{Lighthouse} all'interno del browser Google Chrome.

\subsubsection{Ambienti di test}
Il sito è stato testato su ambiente Linux, Windows e MacOS, sui browser Firefox, Chromium, Google Chrome, Microsoft Edge.\\
Dopo aver deciso di utilizzare il formato \textit{webp} per le immagini per questioni di performance, abbiamo abbandonato il supporto alle varie versioni di Internet Explorer. Nonostante ciò, vista la nostra utenza è molto improbabile avere accessi da un browser di quella famiglia.\\

\subsubsection{Risultati Lighthouse}
Tramite i report generati da Lighthouse abbiamo cercato di tenere al massimo i risultati. Accessibilità, Best Practices e SEO risultano 100 in tutte le pagine.\\
Alcuni punti vengono persi in performance. La prima miglioria è stata quella di utilizzare un formato più leggero per le immagini. Una proposta di Lighthouse è quella di inserire inline alcune parti di CSS e JS, ma questo andrebbe contro il concetto di separazione di struttura, presentazione e comportamento.\\
Per essere il più possibile \textit{mobile friendly} abbiamo cercato di dare una misura adatta al touch alla maggior parte dei bottoni presenti nel nostro sito.
