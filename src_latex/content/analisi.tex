\section{Analisi}
\subsection{Descrizione testuale}

Ci si propone di realizzare un sito web che consensta agli studenti dell'Università di Padova iscritti al Dipartimento di Matematica di accedere ad informazioni relative alla propria carriera di studi.\\
Inoltre si vuole mettere a disposizione un punto dove recuperare informazioni velocemente per gestire la vita al di fuori delle lezioni come ad 
esempio indicazioni per aule studio e punti ristoro.\\
Questo sito punta a centralizzare le informazioni che si posso recuperare, spesso con fatica, dai vari siti di ateneo e dei dipartimenti.\\
Proprio per questo particolare attenzione viene posta sull'esperienza da dispotivo mobile, dato che gli studenti sono molto spesso in movimento e sfruttano lo smartphone per risolvere velocemente dei dubbi.

Gli argomenti che di cui gli studenti hanno più dubbi, stando a quanto si nota dai vari gruppi universitari, sono:
\begin{itemize}
    \item Informazioni e consigli sui vari corsi e come affrontarli al meglio. Questi possono essere semplici dubbi riguardanti gli orari delle lezioni, quali mail utilizzare per contattare il docente.
    %Ad esempio inizio e fine lezioni, Consigli di studio
    \item Informazioni riguardanti edifici e aule. Oltre al cercare i vari indirizzi di aule di lezione e aule studio, spesso si vuole sapere se sono anche fornite di connessione e, ad esempio, prese di corrente. Anche le informazioni per l'accesso sono molto richieste, ad esempio se è necessaria una prenotazione o meno. 
    \item Utilizzo dei laboratori. Ogni dipartimento ha i suoi laboratori e l'accesso ad essi, sia esso fisico ma anche a livello di account, causa spesso dubbi agli studenti.
    \item Utilizzo delle reti wifi. La necessità di una rete buona e stabile è molto sentita da ogni studente, per cui è utile avere una guida su come connettersi alle varie offerte dall'ateneo.   
    \item Partecipazione allo stage e consegna della tesi. Sono gli ultimi passi nella carriera universitaria ed è difficile completarli a causa di vari dettagli burocratici sparsi per i vari siti di dipartimento. Una raccolta chiara e semplice di queste informazioni può aiutare tutti gli studenti interessati.
    %\item Come immatricolarsi %dove trovare i luoghi per i TOLC
    % Li vogliamo sul serio i non immatricolati?
    \item Quali sono i documenti e gli strumenti utili da avere sempre a portata di mano: il badge universitario, il QR per la mensa e le varie applicazioni UniPD.
    % GP QR mensa Badge, Badge fisico
    \item Tutta la documentazione e i link utili per il progetto Erasmus
    % \item Altre cose misc
\end{itemize}

\subsection{Analisi del target di utenza}

Come dicevamo gli utenti a cui ci rivolgiamo sono principalmente gli studenti del CDL Informatica. Per via della vicinanza fisica ai Dipartimenti di Matematica, Fisica e Ingegneria anche i loro studenti potrebbero voler accedere ad alcune delle informazioni contenute nel sito.

\subsubsection{Studenti di Informatica}

Gli studenti di informatica hanno alcune chiare pecularità, hanno ben chiara l'utilità degli aggiornamenti ed in generale sono sempre molto curiosi di provare nuove funzionalità offerte dai software. Da questo possiamo dedurre che il browser sarà sempre aggiornato all'ultima versione o quasi.

% Target principale
Questi utenti sono interessati alla totalità del contenuto del sito.

%Da scrivere meglio
Probabilmente usufruiranno dei contenuti da mobile perché ad esempio avranno bisogno dei contenuti poco prima di dover accedere in mensa o nelle aule studio.

Inoltre gli studenti sono abituati ad un utilizzo di internet agile e veloce:

La loro primaria necessità è quella di effettuare ricerche in maniera veloce, 

secondariamente, tenderanno a interagire gli uni con gli altri, magari discutendo sugli argomenti cercati.

%Relativamente all'analogia della pesca questi utenti sono: -una trappola per aragoste, pesca precisa e una boa di segnalazione

\subsubsection{Persone interessate ad immatricolarsi ad Informatica}
Sono persone curiose che hanno tendenzialmente abbastenza tempo dato che stanno prendendo una decisione importante e possono permettersi di sedersi davanti ad un computer e leggere con attenzione le caratteristiche di cosa si andrà a fare.

%Relativamente all'analogia della pesca questi utenti sono : - pesca a strascico, trappola per aragoste e boa di segnalazione.

\subsubsection{Studenti di altri corsi di laurea}
Questi utenti possiamo paragonarli in molti aspetti agli studenti di informatica, sono molto aggiornati, non sempre come quelli di informatica ma utilizzano comunque le ultime tecnologie. Fanno molto un utilizzo dei contenuti in mobilità poichè il macro-ambiente è lo stesso degli studenti di informatica.

%Relaticamente all'analogia della pesca sono: -una trappola per aragoste e una boa di segnalazione.

\subsection{Conclusioni}

Da questo si può evincere che l'utenza target di questo sito ha browser aggiornati in grado di supportare gli utlimi update degli standard HTML5 e CSS3, quindi risulterebbe poco sensato imporsi limitiazioni sull'utilizzo di nuove feauture offerte dagli standard.

\subsubsection{Ricerca}
L'utente deve poter effettuare la ricerca delle informazioni a cui è interessato all'interno di tutto il sito.
\subsubsection{Login}
L'utente può iscriversi in modo da salvare le sue preferenze. Ad esempio a quale corso è iscritto.
\subsubsection{Dashboard utente}
Dalla dashboard l'utente registrato può selezionare le informazioni alle quali accede più frequentemente.
\subsubsection{Sezione pre immatricolati}
Un utente che non fa ancora parte dell'università può visualizzare alcune informazioni utili.
\subsubsection{Stampa microguide}
Un utente può salvarsi tramite stampa alcune pagine in modo da poterle utilizzare offline. 