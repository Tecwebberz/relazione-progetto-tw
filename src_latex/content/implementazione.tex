\section{Implementazione}

\subsection{Back-end}

\subsubsection{Meccanismo di templating}
È stato deciso di astrarre il meccanismo di template tramite le classi PHP
\textit{TemplateEngine} e \textit{Template} presenti nel file 
\textit{lib\slshape template.engine.php}, il motivo di questa scelta è stato quello di
centralizzare la gestione dei template in modo da permettere a tutti gli
sviluppatori di lavorare su template scritti da altri senza problemi, ma
soprattutto quello di non mostrare mai agli utenti pagine in cui non sono stati
sostituiti tutti i pattern così da non generare mai pagine HTML invalide che
potrebbero causare problemi di inacessibilità del sito.

\subsection{Connessione al database}
Si è creata la classe PHP \textit{DatabaseLayer} presente in
\textit{lib/databaselayer.php} per astrarre tutte le operazioni sia di connessione/
disconnessione al db e di query:
\begin{itemize}
    \item Connessione: la connessione al db può operare in due modi: modalità non
        persistente in cui la connessione è aperta prima della query e chiusa
        subito dopo così da rendere meno verboso ed error prone tutte le pagine
        PHP che utilizzano solo una query al db; modalità persistente tramite
        il metodo \textit{persist()} che permette di gestite la connessione
        manualemente e si usa quando la pagina esegure più di una query.
    \item Query: si è deciso di astrarre il meccanismo di query per forzare sempre
        l'utilizzo dei prepared statement così da prevenire iniezioni SQL.
\end{itemize}

\subsection{Meccanismo ORM}
Si è creata una serie di classi presenti in \textit{lib\/orm\/} per
astrarre le operazioni delle varie entries nel nel db per garantire una interfaccia
uniforme a tutti i programmatori e semplificare 