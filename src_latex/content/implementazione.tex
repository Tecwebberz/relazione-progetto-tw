\section{Implementazione}

\subsection{Front-end}

\subsubsection{Script di validazione}
Per gestire la validazione lato client dei dati immessi dall'utente è stato scritto un JavaScript che li controlla tramite espressioni regolari. I contenuti dei campi del form vengono validati ogni qualvolta il focus su di loro cambia.\\
Gli errori vengono mostrati all'utente tramite una lista non ordinata che viene appesa alla \textit{label} del campo. Per fare in modo che questo nuovo elemento dinamico abbia un significato di \textit{alert} lo script aggiunge quel ruolo ARIA alla lista.
Questo script gestisce inoltre il \texttt{max} e il \texttt{placeholder} del campo \textit{Anno di immatricolazione} in modo che siano impostati sull'anno attuale.

\subsubsection{Script di modifica recensione}
Un utente può modificare o eliminare una recensione, un admin può eliminare qualsiasi recensione.\\
Se l'utente ha i permessi adatti sotto la recensione compaiono due bottoni. Il bottone di modifica invoca uno script che agisce nel seguente modo:
\begin{enumerate}
    \item Ricava quante modifiche sono attualmente attive per creare poi label uniche.
    \item Comincia a costruire un form di modifica che rispecchia quello di una nuova recensione, estraendo il voto e il testo già presenti.
    \item Nasconde la recensione che si vuole modificare, eliminando direttamente dalla visualizzazione gli elementi presenti, dato che durante la modifica non devono poter essere accessibili.
    \item Se la modifica viene effettuata il form la invia, altrimenti tramite il bottone di annullamento gli elementi precedentemente nascosti vengono resi nuovamente visibili.
\end{enumerate}
Il secondo bottone invece cancella direttamente la recensione presente.

\subsubsection{Script di ordinamento aule}
Nella pagina \textit{aule.php} se il browser supporta le API di geolocalizzazione verrà mostrato un bottone per ordinarle in base alla distanza dall'utente, che verrà poi riportata in un angolo nella card relativa alla singola aula.\\
Si è deciso di non mostrare il bottone se l'oggetto \texttt{navigator} non supporta la funzionalità di geolocalizzazione per non dare la parvenza di escludere parte dell'utenza.\\
Per l'implementazione dei calcoli della distanza su una sfera si è deciso di non usare nessuna libreria per rendere il più performante possibile l'applicazione. Il calcolo viene inoltre effettuato una singola volta per evitare eccessivi consumi di batteria.

\subsubsection{Stampa}
Per permettere agli utenti di salvare offline le informazioni presenti nel nostro sito è stato predisposto un foglio di stile aggiuntivo che modifica e adatta la pagina alla stampa \footnote{Funzione miniguide}, sia digitale in PDF che verso la stampa la carta stampata. La maggior parte dei componenti interattivi vengono nascosti per dare la precedenza al contenuto vero e proprio, il font viene sostituito con uno adatto alla stampa ovvero uno con grazie.\\
Alcune pagine sono state definite non stampabili dato che non contengono alcun contenuto utile offline e volevamo tenere conto anche dell'aspetto ecologico.

\subsection{Back-end}

\subsubsection{Meccanismo di templating}
È stato deciso di astrarre il meccanismo di template tramite le classi PHP \texttt{TemplateEngine} e \texttt{Template} presenti nel file \textit{lib/template.engine.php}, questo per permettere a tutti gli sviluppatori del progetto di lavorare contemporaneamente ed efficientemente ai template scritti da altri senza intoppi.\\
Speciale attenzione è stata posta nel non mostrare mai agli utenti pagine in cui non siano stati sostituiti tutti i componenti, così da non generare mai pagine HTML non valide che potrebbero causare problemi di inacessibilità al sito oppure errori nella presentazione.

\subsubsection{Connessione al database}
Si è creata la classe PHP \texttt{DatabaseLayer} presente in \textit{lib/databaselayer.php} per astrarre tutte le operazioni di connessione e disconnessione al database e di query:
\begin{itemize}
    \item \textbf{Connessione:} la connessione al database può operare in due modi:
    \begin{itemize}
        \item Una modalità non persistente in cui la connessione è aperta prima della query e chiusa subito dopo, così da rendere meno verboso ed \textit{error prone} tutte le pagine PHP che utilizzano solo una query al database.
        \item Una modalità persistente tramite il metodo \texttt{persist()} che permette di gestire la connessione manualemente e si usa quando la pagina deve eseguire più di una query.
    \end{itemize}
    \item \textbf{Query:} si è deciso di astrarre il meccanismo di query per forzare sempre l'utilizzo dei \textit{prepared statement} così da prevenire attacchi informatici quali \textit{SQL Injection}.
\end{itemize}

\subsubsection{Meccanismo ORM}
Si è creata una serie di classi presenti in \textit{lib/orm/} per astrarre le operazioni delle varie \textit{entries} nel nel database per garantire una interfaccia uniforme a tutti i programmatori, rendere estensibile il progetto e semplificare la migrazione ad altre tecnologie per basi di dati.

I \texttt{Service} contenuti in \textit{lib/orm/services/} astraggono le operazioni eseguibili sulle varie tabelle del database e le varie relazioni. Questi oggetti prendono come parametro un'istanza di \texttt{DatabaseLayer}, in particolare \texttt{UserService} ha anche una funzione con cui effettuare l'\textit{hashing} delle password degli utenti per garantire una maggiore sicurezza nello stoccaggio dei dati sensibili degli utenti.

I \texttt{DTO} contenuti in \textit{lib/orm/dto/} invece forniscono le operazioni necessarie per operare sulle singole \textit{entry} delle varie tabelle e le varie operazioni di validazione dei dati lato back-end, ovviamente non memorizzando nell'oggetto tutti i dati sensibili, come ad esempio l'hash delle password per evitare potenziali \textit{leaks}.

\subsubsection{Target dei form}
Dentro la cartella \textit{app} sono presenti tutti i file PHP che si occupano di gestire i target dei form e quindi l'interazione tra utente e i dati messi a disposizione dell'applicativo.\\
Questi file si occupano inoltre di eseguire la sanitizzazione dell'input dell'utente, che esso sia malevolo o inconsapevole. Vengono \textit{strippati} i vari tag HTML e tutto il codice che potrebbe causare problemi al processore dei dati.

\subsubsection{Il database}
Il database, ha il compito di salvare a lungo termine e rendere velocemente fruibili tutti i dati utilizzati  dalle parti dinamiche del sito. Come descritto nei capitoli superiori queste sono principalmente suddivise in: utenti, commenti, aule studio e corsi.
Oltre alle categorie di dati principali, abbiamo memorizzato all'interno del database i collegamenti alle immagini con annessi testi alternativi, atti a garantire l'accessibilità anche con il contenuto dinamico.
La scelta di memorizzare questi ultimi è data dall'iniziale pensiero di implementare una pagina amministratore\footnote{Vedi sezione \ref{sviluppi_futuri}} dalla quale poter gestire senza particolari competenze i contenuti del sito, ad esempio aggiungere nuove aule sutudio o cambiare informazioni dei corsi.

Il database è stato chiaramente sviluppato rispettando la forma norrmale, garantendo quindi che i dati al suo interno siano sempre coerenti.