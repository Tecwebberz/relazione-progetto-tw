\section{Progettazione}

Per la progettazione abbiamo usato una strategia \textit{Mobile First} e \textit{Responsive Web Design}.\\
L'approccio \textit{Mobile First} ci permette di concentrarci sulla modalità principale di accesso del nostro target di utenza, ovvero quella da mobile in movimento. Questo comporta una struttura dell'informazione chiara e di semplice fruizione.\\
Invece l'approccio \textit{Responsive Web Design} si basa sul concetto di \textit{media query}, strumenti che ottimizzano la visualizzazione per specifiche dimensioni di dispositivi e vieport.\\
Dall'analisi dell'utenza possiamo assumere che il sito verrà visitato da browser aggiornati e che quindi supportano le ultime feature degli standard HTML5 e CSS3.\\

\section{Tipi di utente}

\begin{itemize}
    \item \textbf{Utente non registrato}\\ Principale tipologia di utente, potrà solamente fruire del contenuto offerto dal sito e dagli utenti registrati.
    \item \textbf{Utente registrato}\\ Oltre alle funzionalità di base potrà lasciare recensioni, modifcare le sue recensioni precendetemente lasciate ed eliminarle.
    \item \textbf{Amministratore} \\ Oltre alle funzionalità disponibili all'utente registrato potrà eliminare le recensioni di tutti gli utenti registrati.
\end{itemize}


\subsection{Struttura del sito}

Il sito presenta molteplici livelli di profondità:
\begin{itemize}
    \item \textbf{Home: } Vengono illustrate in breve le funzioni principali del sito.
    \item \textbf{Corsi: } Vengono elencati i vari corsi del CdL in informatica suddivisi in base all' anno in cui andrebbero seguiti:
        \begin{itemize}
            \item \textbf{Corso: } \`E la pagina dedicata ad un corso nello specifico su di essa si può lasciare e leggere le varie recensioni oltre ad ottenere informazioni specifiche del corso, quali una breve descrizione, i CFU e contatto del professore che lo organizza.
        \end{itemize}
    \item \textbf{Aule: } Vengono elencate le aule studio vicine ai luoghi frequentati dagli studenti del CdL in informatica accompagnati da una breve descrizione e dalla possibilità di ordinarli per distanza tramite geolocalizzazione:
        \begin{itemize}
            \item \textbf{Aula: } È la pagina dedicata ad un' aula studio nello specifico, essa mette a disposizione delle foto, informazioni sui servizi disponibili e recensioni.
        \end{itemize}
    \item \textbf{FAQ: } \`E disponibile una serie di domande frequentemente poste sui vari canali del CdL in informatica accompaganti da risposte.
\end{itemize}

\subsection{Emotional design}
Durante la progettazione del sito si è deciso di voler trasmettere un senso di apparteneza ed esclusivtà agli utenti iscritti in modo da spingerli a creare contenuto in modo da aiutare tutti gli altri studenti.
Inoltre nella pagina 404 si è deciso di utilizzare una battuta in cui gli studenti che hanno visitato almeno una volta Torre Archimede possono identificarsi.

% https://context.reverso.net/traduzione/inglese-italiano/relatable

\subsubsection{Header}
\subsubsection{Breadcrumb}
Per permette agli utenti una facile navigazione del sito e per evitare la dispersione è stata aggiunta una breadcrumb in ogni pagina visitabile. Sfruttando il tag semantico \textit{<nav>} la breadcrumb è stata sfrutturata tramite una \textit{ordered list} per rispecchiare l'ordine di navigazione del sito. Inolte sono stati aggiunti gli attributi \textit{aria-label="breadcrumb"} per dare un significato aggiuntivo alla \textit{ol} e la pagina attuale identificata come \textit{aria-current="page"} per evitare link circolari. Si è tenuto conto della presenza di \textit{Home} di lingua inglese e inoltre le freccette vengono inserite tramite CSS essendo solamente di presentazione.

\subsubsection{Contenuto}
\subsubsection{Footer}
All'interno del footer sono stati inseriti solamente i nomi dei componenti del gruppo e le sedi.
\subsubsection{Database}

\subsection{Accessibilità}
Tutto lo sviluppo del sito si è svolto tenendo a mente le raccomandazioni dello standard WCAG 2.0

Abbiamo inoltre testato il tutto numerose volte ed in numerose occasioni tramite Lighthouse, software di google fornito all'interno di Google Chrome.

\subsubsection{Separazione tra contenuto, presentazione e struttura}
La separazione tra queste tre parti fondamentali del sito ha permesso di gestire al meglio le possibili categorie di accesso effettuate dai vari utenti. La parte di contenuto è stata sviluppata tramite HTML5 in modo da sfruttare a pieno i tag semantici e le nuove aggiunte dello standard. Tramite CSS sono state poi aggiunte tutte le regole di presentazione per il layout del sito. Il comportamento dinamico del sito è stato sviluppato con il linguaggio Javascript.\\ Sfruttando gli strumenti del W3C, ad esempio il validatore di HTML e CSS, ci siamo accertati di aver rispettato tutte le regole dello standard.

\subsection{Assenza di CSS}
In caso di assenza di CSS, il sito mantiene una struttura comprensibile e veicola tutto il contenuto esclusivamente in HTML5. Per un'utente che utilizza strumenti di accesso facilitato per visitare il sito tutto è comunque raggiungibile.

\subsection{Assenza di JS}
Gli script Javascript sfruttano tutte funzionalità base del linguaggio e quindi non richiedono librerie esterne.
% togliere bottoni modifica recensione se non c'è js attivo

\subsection{Attributi aria}
Sfruttando inoltre i tag \textit{aria} si è cercato di dare maggiore significato a varie componenti della pagina quali la breadcrumb. Nei form invece l'attributo \textit{aria-required} è stato omesso dato che è già presente quello standard HTML nei tag di input.