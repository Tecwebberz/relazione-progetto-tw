\section{Progettazione}

Per la progettazione abbiamo usato una strategia \textit{Mobile First} e \textit{Responsive Web Design}.\\
L'approccio \textit{Mobile First} ci permette di concentrarci sulla modalità principale di accesso del nostro target di utenza, ovvero quella da mobile in movimento. Questo comporta una struttura dell'informazione chiara e di semplice fruizione.\\
Invece l'approccio \textit{Responsive Web Design} si basa sul concetto di \textit{media query}, strumenti che ottimizzano la visualizzazione per specifiche dimensioni di dispositivi e vieport.\\
Dall'analisi dell'utenza possiamo assumere che il sito verrà visitato da browser aggiornati e che quindi supportano le ultime feature degli standard HTML5 e CSS3.\\

\section{Tipi di utente}

\begin{itemize}
    \item \textbf{Utente non registrato}\\ Principale tipologia di utente, potrà solamente fruire del contenuto offerto dal sito e dagli utenti registrati.
    \item \textbf{Utente registrato}\\ Oltre alle funzionalità di base potrà lasciare recensioni, modifcare le sue recensioni precendetemente lasciate ed eliminarle.
    \item \textbf{Amministratore} \\ Oltre alle funzionalità disponibili all'utente registrato potrà eliminare le recensioni di tutti gli utenti registrati.
\end{itemize}


\subsection{Struttura del sito}

Il sito presenta molteplici livelli di profondità:
\begin{itemize}
    \item \textbf{Home: } Vengono illustrate in breve le funzioni principali del sito.
    \item \textbf{Corsi: } Vengono elencati i vari corsi del CdL in informatica suddivisi in base all' anno in cui andrebbero seguiti:
        \begin{itemize}
            \item \textbf{Corso: } \`E la pagina dedicata ad un corso nello specifico su di essa si può lasciare e leggere le varie recensioni oltre ad ottenere informazioni specifiche del corso, quali una breve descrizione, i CFU e contatto del professore che lo organizza.
        \end{itemize}
    \item \textbf{Aule: } Vengono elencate le aule studio vicine ai luoghi frequentati dagli studenti del CdL in informatica accompagnati da una breve descrizione e dalla possibilità di ordinarli per distanza tramite geolocalizzazione:
        \begin{itemize}
            \item \textbf{Aula: } È la pagina dedicata ad un' aula studio nello specifico, essa mette a disposizione delle foto, informazioni sui servizi disponibili e recensioni.
        \end{itemize}
    \item \textbf{FAQ: } \`E disponibile una serie di domande frequentemente poste sui vari canali del CdL in informatica accompaganti da risposte.
\end{itemize}

\subsection{Emotional design}
Durante la progettazione del sito si è deciso di voler trasmettere un senso di apparteneza ed esclusivtà agli utenti iscritti in modo da spingerli a creare contenuto in modo da aiutare tutti gli altri studenti.
Inoltre nella pagina 404 si è deciso di utilizzare una battuta in cui gli studenti che hanno visitato almeno una volta Torre Archimede possono identificarsi.

\subsubsection{Header}
Nell'header è presente il logo del sito identificato da un tag \textit{h1} per indicare la prima intestazione. Poi a seguire una \textit{navbar} strutturata come \textit{unordered list} che racchiude le varie pagine del sito. Per evitare loop di link, quello della pagina attuale viene sempre rimosso e reso differente dagli altri.

\subsubsection{Breadcrumb}
Per permette agli utenti agevolare la visita del sito e per evitare la dispersione è stata aggiunta una breadcrumb in ogni pagina visitabile. Sfruttando il tag semantico \textit{<nav>} e strutturata tramite una \textit{ordered list} per rispecchiare l'ordine di navigazione. Anche in questo elemento la pagina attuale non è un link attivo per evitare loop. Si è tenuto conto della presenza di \textit{Home} di lingua inglese e inoltre le freccette vengono inserite tramite CSS essendo solamente di presentazione.

\subsubsection{Contenuto}
Per ogni pagina si è cercato di mantenere una struttura semplice che permettesse all'utente di rispondere in modo semplice alle tre domande fondamentali:
    \begin{itemize}
        \item \textbf{Dove sono?:} domanda a cui è facile rispondere grazie all'header e specialmente alla breadcrumb.
        \item \textbf{Di cosa si tratta?:} grazie alla struttura semplice e minimale di ogni pagina il contenuto principale salta direttamente all'occhio dell'utente.
        \item \textbf{Dove posso andare?:} anche qui data la gerarchia su cui si basano le pagine del sito i percorsi sono facilmente intuibili dall'utente. Inoltre sempre grazie alla breadcrumb è possibile farsi un'idea di come risalire le pagine.
    \end{itemize}
Vediamo per ogni pagina come è strutturato il contenuto:
\begin{itemize}
    \item \textbf{Home} In questa pagina è presente una semplice introduzione allo scopo del sito e tutte le possibili pagine raggiungibili si trovano nell'header
    \item \textbf{Corsi} Qui sono presenti, suddivisi tramite uno schema esatto (per anno) tramite una \textit{ordered list} che poi si suddivide a sua volta in sottoliste \textit{unordered} dove sono presenti bottoni che agiscono da link verso le specifiche pagine del corso.
    \begin{itemize}
        \item \textbf{Corso} La pagina del singolo corso mantiene una struttura minimale con all'interno una lista \textit{non ordinata} che contiene le informazioni principali quali referente e contatti, anno e semestre di erogazione. Inoltre è presente il form per inviare la recensione e quelle già presenti, con la media calcolata.
    \end{itemize}
    \item \textbf{Aule} La pagina delle aule presenta una lista di aule \textit{ordered}. Questo perchè è già predisposta per essere ordinata in base alla distanza dall'utente tramite uno script Javascript. Tutte le aule poi sono strutturate in una \textit{css grid} che racchiude le \textit{cards} delle singole aule.
    \begin{itemize}
        \item \textbf{Aula} La pagina dell'aula contiene una galleria di immagini rappresentative. Sono contenute tutte le informazioni interessanti dell'aula quali posti disponibili, posizione e presenza o meno di connessione e possibilità di utilizzare prese di corrente. Anche in queste pagine è possibile lasciare e visualizzare le recensioni.
    \end{itemize}
    \item \textbf{FAQ} La pagina delle \textit{FAQ} contiene le \textit{Frequently Asked Question}. Queste sono rappresentate tramite una \textit{definition list}. Questa pagina è principalmente statica in quanto queste risposte sono già ben definite.
    \item \textit{Accedi e Registrati} Le due pagine che permettono all'utente di entrare nella community 
\end{itemize}
\subsubsection{Footer}
All'interno del footer sono stati inseriti solamente i nomi dei componenti del gruppo e le sedi.
\subsubsection{Database}

\subsection{Accessibilità}
Tutto lo sviluppo del sito si è svolto tenendo a mente le raccomandazioni dello standard WCAG 2.0

Abbiamo inoltre testato il tutto numerose volte ed in numerose occasioni tramite Lighthouse, software di google fornito all'interno di Google Chrome.

\subsubsection{Separazione tra contenuto, presentazione e struttura}
La separazione tra queste tre parti fondamentali del sito ha permesso di gestire al meglio le possibili categorie di accesso effettuate dai vari utenti. La parte di contenuto è stata sviluppata tramite HTML5 in modo da sfruttare a pieno i tag semantici e le nuove aggiunte dello standard. Tramite CSS sono state poi aggiunte tutte le regole di presentazione per il layout del sito. Il comportamento dinamico del sito è stato sviluppato con il linguaggio Javascript.\\ Sfruttando gli strumenti del W3C, ad esempio il validatore di HTML e CSS, ci siamo accertati di aver rispettato tutte le regole dello standard.

\subsection{Assenza di CSS}
In caso di assenza di CSS, il sito mantiene una struttura comprensibile e veicola tutto il contenuto esclusivamente in HTML5. Per un'utente che utilizza strumenti di accesso facilitato per visitare il sito tutto è comunque raggiungibile.

\subsection{Assenza di JS}
Gli script Javascript sfruttano tutte funzionalità base del linguaggio e quindi non richiedono librerie esterne.

\subsection{Attributi aria}
Sfruttando inoltre i tag \textit{aria} si è cercato di dare maggiore significato a varie componenti della pagina quali la breadcrumb. Nei form invece l'attributo \textit{aria-required} è stato omesso dato che è già presente quello standard HTML nei tag di input.

\subsection{Suddivisione del lavoro}
Purtroppo a causa di foza maggiore dovute allo status di salute di uno dei membri del gruppo orginale il progetto è stato svolto da 3 persone suddividendosi il lavoro nei seguenti modi:
\begin{itemize}
    \item \textbf{Alessio Ferrarini:}
        \begin{itemize}
            \item Progettazione del sito web.
            \item Sviluppo dei componenti simil ORM (Object relational mapping) in PHP.
            \item Sviluppo del componentPHPer astrarre la sosituzione nei template html.
            \item RE\item Sviluppo degli script php che si occupano dell' inserimento e rimozione dei dati tramite 
            l'interealizzazione delfa realazione.iaccia ORM.
            \item Sviluppo script javascript per l'ordinamento delle aule tramite API di geolocalizzazione.
        \end{itemize}
    \item \textbf{Alessandro Massarenti:}
        \begin{itemize}
            \item Progettazione del sito web.
            \item Design dell' aspetto grafico del sito.
            \item Svilupo  delladegli stili CS
            \item Gestione e creazione delle immagini p.esenti nel sitoS
            \item Realizzazione della realazione..
            \item Sviluppo base di dati.
        \end{itemize}
    \item \textbf{Elia Pasquali:}
        \begin{itemize}
            \item Progettazione del sito wHTML            \item Design dell' aspetto gJavaScript sito.
iut del form            \item Sviluppo deJavaScript javascript dedicati alla comparsa/scomparsa di componenti i
            \item Inserimento dei componenti HTML con attributi aria dedicati all' accessibilità.
            \item Realizzazione della realazione.nterattivi. avascript dedicati alla validazione  egli in         \item Sviluppo base di dati.
        \end{itemize}
t\end{itemize}
Nonostant questa suddivisione tt i mebri mbri hanno collaborato in tutti gli aspetti della realizzazione del sito soprattutto per quanto riguarda l'accessibiltà di esso.ta suddivisione tt i mebri 