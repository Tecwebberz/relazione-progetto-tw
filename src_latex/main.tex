\documentclass[a4paper, 11pt]{article}
\usepackage{hyperref}
\usepackage{geometry}
\usepackage{longtable}
\usepackage[table,xcdraw]{xcolor}
\usepackage{float}
 \geometry{
 a4paper,
 left=25mm,
 right=25mm,
 top=20mm,
 bottom=20mm,
 }

\setlength{\parskip}{1em}
\setlength{\parindent}{0pt}

\title{Progetto TecWeb \\ UniHelp@Unipd}

%Plausibile titolo: UniHelp?

\author{Alessio, Alessandro, Elia}
\date{Novembre 2021}


\begin{document}

\begin{center}
	\textbf{\Huge{Unihelp@Unipd}}\\
\end{center}

\vspace{1.5cm}

\begin{center}
	\textbf{\huge{Relazione progetto Tecnologie Web}}\\[0.2cm]
	\Large{Anno 2021-2022}
\end{center}

\vspace{5pt}

\begin{center}
	\textbf{\Large{Gruppo}}
    \begin{table}[H]
        \hspace{3.5cm}
        \renewcommand{\arraystretch}{1.4}
        \begin{tabular}{l | l}
            \textbf{Componenti} & Alessio Ferrarini 1223860\\
            & Alessandro Massarenti 1204684\\
            & Elia Pasquali 1225412\\
        \end{tabular}
    \end{table}
\end{center}

\hspace{5pt}

\begin{center}
	\textbf{\Large{Informazioni sul sito}}\\
	\textbf{Login consegna:} elpasqua \\
	\textbf{Indirizzo sito web:} \url{https://tecweb.studenti.math.unipd.it/elpasqua}\\
	\textbf{Email referente gruppo:} alessandro.massarenti@studenti.unipd.it\\
    \vspace{1cm}
    \textbf{Accessi interni al sito}
    \begin{longtable}{|l|l|l|}
        \hline
        \rowcolor[HTML]{9B0014}
        {\color[HTML]{FFFFFF} Ruolo} & {\color[HTML]{FFFFFF} Username} & {\color[HTML]{FFFFFF} Password} \\ \hline
        Utente         & user     & user     \\ \hline
        Amministratore & admin    & admin \\ \hline
    \end{longtable}
\end{center}

\pagebreak
\tableofcontents
\pagebreak

\section{Analisi}
\subsection{Descrizione testuale}

Ci si propone di realizzare un sito web che consensta agli studenti dell'Università di Padova iscritti al Corso di Laurea in Informatica di accedere ad informazioni relative alla propria permanenza nei locali dell'ateneo e ai corsi.\\
Inoltre si vuole mettere a disposizione un punto dove recuperare informazioni velocemente per gestire la vita al di fuori delle lezioni come ad esempio indicazioni per aule studio.\\
Questo sito punta a centralizzare le informazioni che si posso recuperare, spesso con fatica, dai vari siti di ateneo e dei dipartimenti.\\
Proprio per questo particolare attenzione viene posta sull'esperienza da dispotivo mobile, dato che gli studenti sono molto spesso in movimento e sfruttano lo smartphone per risolvere velocemente dei dubbi.\\
Gli argomenti di cui gli studenti hanno più dubbi, stando a quanto si nota dai vari gruppi universitari, sono:
\begin{itemize}
    \item Informazioni e consigli sui vari corsi e come affrontarli al meglio.
    \item Informazioni riguardanti edifici e aule.\\ Indirizzi, informazioni sulla presenza di reti wifi e prese di corrente, senza dimenticare le modalità di accesso.
    \item Utilizzo delle reti wifi. \\ La necessità di una rete buona e stabile è molto sentita da ogni studente, per cui è utile avere una guida su come connettersi alle varie offerte dall'ateneo.   
    \item Partecipazione allo stage e consegna della tesi. \\ Sono gli ultimi passi della carriere universitaria e le istruzioni sono sparse e nascoste nei vari siti dell'ateneo. L'idea è di centralizzarle e renderle di facile comprensione.
    \item Quali sono i documenti e gli strumenti utili da avere sempre a portata di mano: il badge universitario, il QR per la mensa e le varie applicazioni UniPD.
\end{itemize}

\subsection{Analisi del target di utenza}

L'utenza sarà principalmente formata da studenti del CdL in Informatica. Data la vicinanza geografica con i dipartimenti di Matematica, Fisica e Ingegneria nulla vieta anche ai loro studenti di visitare il nostro sito per ottenere alcune informazioni utili.\\
Dall'analisi appena fatta quindi possiamo aspettarci degli utenti consapevoli nell'utilizzo dei loro strumenti, con aggiornamenti frequenti e attuali, curiosi nel provare nuove funzionalità offerte dai software. Per questo il nostro sito potrà utilizzare senza troppi rischi nuovi standard.

Il sito permette a tutte le tipologie di ricerca di essere effettuate con
efficacia:
\begin{itemize}
    \item \textbf{Tiro perfetto:} Grazie ad uno schema organizzativo esatto
        l'utente può raggiungere l'informazione che cerca al primo colpo.
    \item \textbf{Trappola per aragoste: } Grazie alla suddivisione in
        sottocategorie sia dei corsi che delle aule studio l'utente può già
        farsi una idea del contenuto disponibile sul sito grazie alle
        descrizioni.
    \item \textbf{Pesca con la rete:} La struttura del sito è particolarmente
        semplice ed intuitiva, questo permette agli utenti di esporare la
        totalita del sito senza perdere contenuti o entrare in stato di
        confusione.
    \item \textbf{Boa di segnalazione:} Tutte le pagine visibi all'utente
        utilizzano parametri passaggio di parametri tramite metodo
        \textit{POST} quindi permettendo di salvare le pagine relative a
        qualsiasi contenuto.
\end{itemize}

\subsection{Servizi offerti}
\begin{itemize}
    \item \textbf{Login} \\ L'utente può iscriversi per commentare aule studio e corsi per contribuire con le proprie conoscenze ad una globale e distribuita;
    \item \textbf{Recensioni} \\L'utente deve poter leggere e creare recensioni dei corsi erogati nel Corso di Informatica e delle aule studio;
    \item \textbf{Stampa microguide} \\ Un utente può salvarsi tramite stampa alcune pagine in modo da poterle visualizzare offline;
    \item \textbf{Aule studio più vicine} \\ L'utente potrà trovare le aule studio più vicine a sé tramite la geolocalizzazione.
\end{itemize}

\section{Progettazione}

Per la progettazione abbiamo usato una strategia \textit{Mobile First} e \textit{Responsive Web Design}.\\
L'approccio \textit{Mobile First} ci permette di concentrarci sulla modalità principale di accesso del nostro target di utenza, ovvero quella da mobile in movimento. Questo comporta una struttura dell'informazione chiara e di semplice fruizione.\\
Invece l'approccio \textit{Responsive Web Design} si basa sul concetto di \textit{media query}, strumenti che ottimizzano la visualizzazione per specifiche dimensioni di dispositivi e vieport.\\
Dall'analisi dell'utenza possiamo assumere che il sito verrà visitato da browser aggiornati e che quindi supportano le ultime feature degli standard HTML5 e CSS3.\\

\subsection{Tipi di utente}

\begin{itemize}
    \item \textbf{Utente non registrato:} principale tipologia di utente, potrà solamente fruire del contenuto offerto dal sito e dagli utenti registrati.
    \item \textbf{Utente registrato:} oltre alle funzionalità di base potrà lasciare recensioni, modifcare le sue recensioni precendetemente lasciate ed eliminarle.
    \item \textbf{Amministratore:} oltre alle funzionalità disponibili all'utente registrato potrà eliminare le recensioni di tutti gli utenti registrati.
\end{itemize}

\subsection{Struttura del sito}

Il sito presenta molteplici livelli di profondità:
\begin{itemize}
    \item \textbf{Home: } Vengono illustrate in breve le funzioni principali del sito.
    \item \textbf{Corsi: } Vengono elencati i vari corsi del CdL in informatica suddivisi in base all' anno in cui andrebbero seguiti:
        \begin{itemize}
            \item \textbf{Corso: } È la pagina dedicata ad un corso nello specifico su di essa si può lasciare e leggere le varie recensioni oltre ad ottenere informazioni specifiche del corso, quali una breve descrizione, i CFU e contatto del professore che lo organizza.
        \end{itemize}
    \item \textbf{Aule: } Vengono elencate le aule studio vicine ai luoghi frequentati dagli studenti del CdL in informatica accompagnati da una breve descrizione e dalla possibilità di ordinarli per distanza tramite geolocalizzazione:
        \begin{itemize}
            \item \textbf{Aula: } È la pagina dedicata ad un' aula studio nello specifico, essa mette a disposizione delle foto, informazioni sui servizi disponibili e recensioni.
        \end{itemize}
    \item \textbf{FAQ: } È disponibile una serie di domande frequentemente poste sui vari canali del CdL in informatica accompaganti da risposte.
\end{itemize}

\subsubsection{Header}
Nell'header è presente il logo del sito identificato da un tag \textit{h1} per indicare la prima intestazione. Poi a seguire una \textit{navbar} strutturata come \textit{unordered list} che racchiude le varie pagine del sito. Per evitare loop di link, quello della pagina attuale viene sempre rimosso e reso differente dagli altri.

\subsubsection{Breadcrumb}
Per permette agli utenti agevolare la visita del sito e per evitare la dispersione è stata aggiunta una breadcrumb in ogni pagina visitabile. Sfruttando il tag semantico \textit{<nav>} e strutturata tramite una \textit{ordered list} per rispecchiare l'ordine di navigazione. Anche in questo elemento la pagina attuale non è un link attivo per evitare loop. Si è tenuto conto della presenza di \textit{Home} di lingua inglese e inoltre le freccette vengono inserite tramite CSS essendo solamente di presentazione.

\subsubsection{Contenuto}
Per ogni pagina si è cercato di mantenere una struttura semplice che permettesse all'utente di rispondere in modo semplice alle tre domande fondamentali:
    \begin{itemize}
        \item \textbf{Dove sono?:} domanda a cui è facile rispondere grazie all'header e specialmente alla breadcrumb.
        \item \textbf{Di cosa si tratta?:} grazie alla struttura semplice e minimale di ogni pagina il contenuto principale salta direttamente all'occhio dell'utente.
        \item \textbf{Dove posso andare?:} anche qui data la gerarchia su cui si basano le pagine del sito i percorsi sono facilmente intuibili dall'utente. Inoltre sempre grazie alla breadcrumb è possibile farsi un'idea di come risalire le pagine.
    \end{itemize}
Vediamo per ogni pagina come è strutturato il contenuto:
\begin{itemize}
    \item \textbf{Home} In questa pagina è presente una semplice introduzione allo scopo del sito e tutte le possibili pagine raggiungibili si trovano nell'header
    \item \textbf{Corsi} Qui sono presenti, suddivisi tramite uno schema esatto (per anno) tramite una \textit{ordered list} che poi si suddivide a sua volta in sottoliste \textit{unordered} dove sono presenti bottoni che agiscono da link verso le specifiche pagine del corso.
    \begin{itemize}
        \item \textbf{Corso} La pagina del singolo corso mantiene una struttura minimale con all'interno una lista \textit{non ordinata} che contiene le informazioni principali quali referente e contatti, anno e semestre di erogazione. Inoltre è presente il form per inviare la recensione e quelle già presenti, con la media calcolata.
    \end{itemize}
    \item \textbf{Aule} La pagina delle aule presenta una lista di aule \textit{ordered}. Questo perchè è già predisposta per essere ordinata in base alla distanza dall'utente tramite uno script Javascript. Tutte le aule poi sono strutturate in una \textit{css grid} che racchiude le \textit{cards} delle singole aule.
    \begin{itemize}
        \item \textbf{Aula} La pagina dell'aula contiene una galleria di immagini rappresentative. Sono contenute tutte le informazioni interessanti dell'aula quali posti disponibili, posizione e presenza o meno di connessione e possibilità di utilizzare prese di corrente. Anche in queste pagine è possibile lasciare e visualizzare le recensioni.
    \end{itemize}
    \item \textbf{FAQ} La pagina delle \textit{FAQ} contiene le \textit{Frequently Asked Question}. Queste sono rappresentate tramite una \textit{definition list}. Questa pagina è principalmente statica in quanto queste risposte sono già ben definite.
    \item \textit{Accedi e Registrati} Le due pagine che permettono all'utente di entrare nella community tramite due \textit{form} che sfruttano a pieno i controlli offerti dall'\textit{input HTML5} e i dati inviati vengono poi controllati via Javascript lato client e via PHP lato server.
\end{itemize}
Sono presenti due pagine aggiuntive non direttamente raggiungibili:
\begin{itemize}
    \item \textbf{Registrazione avvenuta:} serve per rendere consapevole l'utente di aver completato la registrazione e che questa è andata a buon fine.
    \item \textbf{Errore 404 e 500:} cercano di aiutare l'utente che per errore si è perso portandolo su una pagina strutturata in modo da reindizzarlo verso il nostro sito.
\end{itemize}

\subsubsection{Footer}
All'interno del footer sono stati inseriti solamente i nomi dei componenti del gruppo e le sedi.

\subsubsection{Database}

\subsection{Emotional design}
Durante la progettazione del sito si è deciso di voler trasmettere un senso di apparteneza ed esclusivtà agli utenti iscritti in modo da spingerli a creare contenuto per aiutare tutti la communitiy.
Inoltre nelle pagina dell'errore 404 è deciso di utilizzare una battuta in cui gli studenti che hanno visitato almeno una volta Torre Archimede possono identificarsi. Invece per la pagina che gestisce gli errori 500 abbiamo puntato di nuovo su una battuta.

\subsection{Accessibilità}
Tutto lo sviluppo del sito si è svolto tenendo a mente le raccomandazioni dello standard WCAG 2.0.\\ Abbiamo inoltre testato il tutto numerose volte tramite Lighthouse, software di google fornito all'interno di Google Chrome.

\subsubsection{Separazione tra contenuto, presentazione e struttura}
La separazione tra queste tre parti fondamentali del sito ha permesso di gestire al meglio le possibili categorie di accesso effettuate dai vari utenti. La parte di contenuto è stata sviluppata tramite HTML5 in modo da sfruttare a pieno i tag semantici e le nuove aggiunte dello standard. Tramite CSS sono state poi aggiunte tutte le regole di presentazione per il layout del sito. Il comportamento dinamico del sito è stato sviluppato con il linguaggio Javascript.\\ Sfruttando gli strumenti del W3C, ad esempio il validatore di HTML e CSS, ci siamo accertati di aver rispettato tutte le regole dello standard.

\subsection{Assenza di CSS}
In caso di assenza di CSS, il sito mantiene una struttura comprensibile e veicola tutto il contenuto esclusivamente in HTML5. Per un'utente che utilizza strumenti di accesso facilitato per visitare il sito tutto è comunque raggiungibile.

\subsection{Assenza di JS}
Gli script Javascript sfruttano tutte funzionalità base del linguaggio e quindi non richiedono librerie esterne.

\subsection{Attributi aria}
Sfruttando inoltre i tag \textit{aria} si è cercato di dare maggiore significato a varie componenti della pagina quali la breadcrumb. Nei form invece l'attributo \textit{aria-required} è stato omesso dato che è già presente quello standard HTML nei tag di input.

\subsection{Suddivisione del lavoro}
Purtroppo a causa di foza maggiore dovute allo stato di salute di uno dei membri del gruppo orginale, il progetto è stato svolto da 3 persone. Ci siamo suddivisi il lavoro nel seguente modo.

\begin{itemize}
    \item \textbf{Alessio Ferrarini:}
        \begin{itemize}
            \item Progettazione del sito web;
            \item Sviluppo dei componenti simil ORM (Object relational mapping) in PHP;
            \item Sviluppo del componente PHP per astrarre la sosituzione nei template HTML;
            \item Sviluppo degli script PHP che si occupano dell'inserimento e rimozione dei dati tramite l'interfaccia ORM;
            \item Sviluppo script javascript per l'ordinamento delle aule tramite API di geolocalizzazione;
            \item Stesura della relazione;
        \end{itemize}
    \item \textbf{Alessandro Massarenti:}
        \begin{itemize}
            \item Progettazione del sito web;
            \item Design dell'aspetto grafico del sito;
            \item Sviluppo dei fogli di stile CSS;
            \item Realizzazione ed impaginazione delle pagine HTML;
            \item Sviluppo della base di dati;
            \item Gestione e creazione ed elaborazione delle immagini presenti nel sito;
            \item Stesura della relazione;
        \end{itemize}
    \item \textbf{Elia Pasquali:}
        \begin{itemize}
            \item Progettazione del sito web;
            \item Sviluppo del JavaScript dedicato alla comparsa/scomparsa di elementi;
            \item Inserimento dei componenti HTML con attributi aria dedicati all'accessibilità;
            \item Stesura della realazione;
            \item Realizzazione del codice JavaScript dedicato alla validazione dei form in maniera accessibile;
        \end{itemize}
\end{itemize}

Nostante questa suddivisione, tutti i mebri membri hanno collaborato in tutti gli aspetti della realizzazione del sito soprattutto per quanto riguarda l'accessibiltà di esso.
\section{Implementazione}

\subsection{Back-end}

\subsubsection{Meccanismo di templating}
È stato deciso di astrarre il meccanismo di template tramite le classi PHP
\textit{TemplateEngine} e \textit{Template} presenti nel file 
\textit{lib\slshape template.engine.php}, il motivo di questa scelta è stato
quello di centralizzare la gestione dei template in modo da permettere a tutti
gli sviluppatori di lavorare su template scritti da altri senza problemi, ma
soprattutto quello di non mostrare mai agli utenti pagine in cui non sono stati
sostituiti tutti i pattern così da non generare mai pagine HTML invalide che
potrebbero causare problemi di inacessibilità del sito.

\subsubsection{Connessione al database}
Si è creata la classe PHP \textit{DatabaseLayer} presente in
\textit{lib/databaselayer.php} per astrarre tutte le operazioni sia di
connessione/disconnessione al db e di query:
\begin{itemize}
    \item Connessione: la connessione al db può operare in due modi: modalità
        non persistente in cui la connessione è aperta prima della query e
        chiusa subito dopo così da rendere meno verboso ed error prone tutte le
        pagine PHP che utilizzano solo una query al db; modalità persistente
        tramite il metodo \textit{persist()} che permette di gestite la
        connessione manualemente e si usa quando la pagina esegure più di una
        query.
    \item Query: si è deciso di astrarre il meccanismo di query per forzare
        sempre l'utilizzo dei prepared statement così da prevenire iniezioni
        SQL.
\end{itemize}

\subsubsection{Meccanismo ORM}
Si è creata una serie di classi presenti in \textit{lib\slshape orm\slshape}
per astrarre le operazioni delle varie entries nel nel db per garantire una
interfaccia uniforme a tutti i programmatori e semplificare la migrazione ad
altre tecnologie per basi di dati.\\

I \textit{Service} contenuti in \textit{lib/orm/services/}
astraggono le operazioni eseguibili sulle varie tabelle del db e le varie
relazioni esse predono come parametro una istanza di \textit{DatabaseLayer}
(\textit{UserService} anche una funzione con cui effettuare l'hash sulle
password degli utenti).\\

I \textit{DTO} contenuti in \textit{lib/orm/dto/} invece forniscono le
operazioni necessarie per operare sulle singole entry delle varie tabelle e
le varie operazioni di validazione dei dati lato back-end ovviamente
togliendo tutti i dati sensibili come hash delle password per evitare
potenziali leaks.

\subsubsection{Target dei form}
Dentro la cartella \textit{app} sono presenti tutti i file php che si occupano
di gestire i target dei form e quindi l'interazione tra utente e i dati messi
a disposizione dell' applicativo.

\subsection{Frontend}

\subsubsection{Ordinamento delle aule in base alla distanza}
Nella pagina \textit{aule.php} se il browser dell'utente supporta le api di
geolocalizzazione verrà mostrto un bottone per ordinare le varie aule in base
alla distanza che verrà anche riportata nella card.\\

Si è deciso di non mostrare il bottone se l'oggetto \textit{navigator} non
supporta la funzionalità di geolocalizzazione per non dare la parvenza di
escludere parte dell'utenza.\\

Per l'implementazione anche calcoli della distanza su una sfera si è deciso di
non usare nessuna libreria per rendere il più performante possibile
l'applicazione.

\section{Sviluppi futuri}
Purtroppo a causa dell'assenza di un componente del gruppo non siamo riusciti a portare a termine tutte le funzionalità che avevamo programmato ad inizio progetto. Inizialmente avevamo infatti immaginito di permettere all'utente amministratore di inserire corsi e aule studio. Infatti il codice per l'inserimento e la generazione delle pagine dedicate ad essi è già totalmente dinamico e pronto per eventuali estensioni future.

\section{Prestazioni e SEO}

Le prestazioni di un sito sono fondamentali per avere un buon posizionamento sui vari motodi di ricerca, quindi si è deciso di:
\begin{itemize}
    \item Minimizzare tutti i file javascript e css.
    \item Sviluppare il codice PHP in \textit{early return style} per diminuire il più possibile i tempi di esecuzione.
    \item Utilizzare standard moderni come \textit{WebP} per ridurre il più possibile il peso delle immagini senza perdere qualità.
    \item Impostato regole di caching più efficaci per i contenuti statici in \textit{htaccess}.
    \item Usato il più possibile i consigli di Google Lighthouse (rispettando sempre la separazione tra contenuto e presentazione,ad  esempio: no inlining di regole css fondamentali), ottenendo sempre 100\% in tutte le categorie tranne prestazioni in cui comunque si ottiene 100\% nella maggioranza delle pagine e un 95\% nelle peggiori.
\end{itemize}

\section{Accessibilità}
Tutto lo sviluppo del sito si è svolto tenendo a mente le raccomandazioni dello standard WCAG 2.0.

\subsubsection{Separazione tra contenuto, presentazione e struttura}
La separazione tra queste tre parti fondamentali del sito ha permesso di gestire al meglio le possibili categorie di accesso effettuate dai vari utenti. La parte di contenuto è stata sviluppata tramite HTML5 in modo da sfruttare a pieno i tag semantici e le nuove aggiunte dello standard. Tramite CSS sono state poi aggiunte tutte le regole di presentazione per il layout del sito. Il comportamento dinamico del sito è stato sviluppato con il linguaggio Javascript.\\ Sfruttando gli strumenti del W3C, ad esempio il validatore di HTML e CSS, ci siamo accertati di aver rispettato tutte le regole dello standard.

\subsubsection{Assenza di CSS}
In caso di assenza di CSS, il sito mantiene una struttura comprensibile e veicola tutto il contenuto esclusivamente in HTML5. Per un'utente che utilizza strumenti di accesso facilitato per visitare il sito tutto è comunque raggiungibile.

\subsubsection{Assenza di JS}
Gli script Javascript sfruttano tutte funzionalità base del linguaggio e quindi non richiedono librerie esterne.

\subsubsection{Attributi ARIA}
Nonostante aver utilizzato il più possibile i tag semantici offerti da HTML5 abbiamo inserito anche alcuni attributi ARIA in modo da rendere più "espressivi" e accessibli.
\begin{itemize}
    \item \textbf{Breadcrumb:} la braedcrumb è stata definita tramite \textit{aria-label} come \textit{breadcrumb}, inoltre la pagina attuale è indicata dall'attributo \textit{aria-page="current"}.
    \item \textbf{Validazione dei form:} lo script di validazione del form aggiunge un elemento nel DOM dinamicamente. Viene quindi indicato tramite dal ruolo \textit{aria-role="alert"} che porta l'attenzione dello screen reader su di esso quando questoviene inserito nella pagina.
    \item \textbf{Bottoni link:} quando non è stato possibile sfruttare tag come \textit{<button>} è stato necessario usare l'attributo \textit{role="button"} su quei link \textit{<a>} che vengono rappresentati come bottoni via CSS.
    \item \textbf{Link aule:} nelle card delle aule è stata aggiunta una \textit{aria-label} dinamica per dare maggior significato al link \textit{Più info}.´
    \item \textbf{Sistema di rating:} la struttura del sistema di rating a stelle fa affidamento al foglio di stile per creare l'animazione delle stelle. In caso di assenza di esso, tutto il form mantiene una struttura accessibile: tutti gli input hanno una label identificativa univoca. Le \textit{<i>} necessarie alle stelle sono state dichiarate \textit{aria-disabled} per gli screenreader.
\end{itemize}
Altri attributi aria come \textit{aria-requried} e simili nei form sono stati omessi dato che sono già presenti quelli nativi di HTML5 che veicolano lo stesso significato.

\subsubsection{Contrasti}
Si è preso come principale colore uno che rimandi all'ateneo e al dipartimento. Da lì tutti gli altri colori sono stati scelti in modo da avere un contrasto che rispetti gli standard AAA del WCAG sfruttando lo strumento online \href{https://app.contrast-finder.org/}{Contrast Finder}. Queste analisi sono state fatte tramite il sito \href{https://contrastchecker.com/}{WCAG - Contrast Checker} e i \textit{Firefox Developer Tools}.

\subsubsection{Tabindex}
Non è sembrato necessario modificare l'ordine dei tabindex manualmente. Per come è stato strutturato il sito questi sono già organizzati in modo adeguato.

\subsubsection{Lingue straniere e termini abbreviati}
Il sito specifica come lingua principale l'italiano. Ogni parola che deve essere letta con pronuncia straniera è stata contrassegnata tramite l'attributo \textit{lang} in un tag di comodo come \textit{span} oppure nel suo tag di appartenza. Ad esempio la semplice parola \textit{Home} nella breadcrumb oppure \textit{stage} nelle FAQ. Inoltre tutte le sigle utilizzate (ad esempio \textit{ESU} ed \textit{ISEE}) sono state inserite all'interno di dei tag \textit{<abbr>}.

\subsection{Strumenti di testing}

Tutte le pagine e i fogli di stile sono state controllate tramite il \href{https://validator.w3.org/}{validatore HTML e CSS} offerto dal W3C.\\
Durante tutto il ciclo di sviluppo la maggior parte dei controlli è stata fatta tramite i \textit{Firefox Developer Tools} e il software \textit{Lighthouse} all'interno del browser Google Chrome.

\subsubsection{Ambienti di test}
Il sito è stato testato su ambiente Linux, Windows e MacOS, sui browser Firefox, Chromium, Google Chrome, Microsoft Edge.\\
Dopo aver deciso di utilizzare il formato \textit{webp} per le immagini per questioni di performance, abbiamo abbandonato il supporto alle varie versioni di Internet Explorer. Nonostante ciò, vista la nostra utenza è molto improbabile avere accessi da un browser di quella famiglia.\\

\subsubsection{Risultati Lighthouse}
Tramite i report generati da Lighthouse abbiamo cercato di tenere al massimo i risultati. Accessibilità, Best Practices e SEO risultano 100 in tutte le pagine.\\
Alcuni punti vengono persi in performance. La prima miglioria è stata quella di utilizzare un formato più leggero per le immagini. Una proposta di Lighthouse è quella di inserire inline alcune parti di CSS e JS, ma questo andrebbe contro il concetto di separazione di struttura, presentazione e comportamento.\\
Per essere il più possibile \textit{mobile friendly} abbiamo cercato di dare una misura adatta al touch alla maggior parte dei bottoni presenti nel nostro sito.


\end{document}
